%%%%%%%%%%%%%%%%%%%%%%%%%%%%%%%%%%%%%%%%%%%%%%%%%%%%%%%%%%%%%%%%%%%%%%%%%%%%%%%%
%2345678901234567890123456789012345678901234567890123456789012345678901234567890
%        1         2         3         4         5         6         7         8

\documentclass[letterpaper, 10 pt, conference]{ieeeconf}  % Comment this line out if you need a4paper

%\documentclass[a4paper, 10pt, conference]{ieeeconf}      % Use this line for a4 paper

\IEEEoverridecommandlockouts                              % This command is only needed if 
                                                          % you want to use the \thanks command

\overrideIEEEmargins                                      % Needed to meet printer requirements.

% See the \addtolength command later in the file to balance the column lengths
% on the last page of the document

% The following packages can be found on http:\\www.ctan.org
%\usepackage{graphics} % for pdf, bitmapped graphics files
%\usepackage{epsfig} % for postscript graphics files
%\usepackage{mathptmx} % assumes new font selection scheme installed
%\usepackage{times} % assumes new font selection scheme installed
%\usepackage{amsmath} % assumes amsmath package installed
%\usepackage{amssymb}  % assumes amsmath package installed

\title{\LARGE \bf
Structure of the set of feasible neural commands for complex motor tasks*
}

\author{TODOFCO(choose authors and order)$^{1}$ and Bernard D. Researcher$^{2}$% <-this % stops a space
\thanks{*This work was supported by TODOFCO(add funding)}% <-this % stops a space
\thanks{$^{1}$Brian Cohn is with the departments of Biomedical Engineering and Computer Science at the University of Southern California Viterbi School of Engineering, Los Angeles, CA 90089, USA
        {\tt\small brianaco@usc.edu}}%
\thanks{$^{2}$Francisco is from TODOFCO
        {\tt\small valero@usc.edu}}%
}


\begin{document}



\maketitle
\thispagestyle{empty}
\pagestyle{empty}


%%%%%%%%%%%%%%%%%%%%%%%%%%%%%%%%%%%%%%%%%%%%%%%%%%%%%%%%%%%%%%%%%%%%%%%%%%%%%%%%
\begin{abstract}
The brain must select its control strategies among an infinite set of possibilities; researchers believe that it must be solving an optimization problem. 
While this set of feasible solutions is infinite and lies in high dimensions, it is bounded by kinematic, neuromuscular, and anatomical constraints, within which the brain must select optimal solutions. That is, the set of feasible activations is well structured. However, to date there is no method to describe and quantify the structure of these high-dimensional solution spaces.
Bounding boxes or dimensionality reduction algorithms do not capture their detailed structure. 
We present a novel approach based on the well-known Hit-and-Run algorithm in computational geometry to extract the structure of the feasible activations capable of producing 50\% of maximal fingertip force in a specific direction.
We use a realistic model of a human index finger with 7 muscles, and 4 DOFs. For a given static force vector at the endpoint, the feasible activation space is a 3D convex polytope, embedded in the 7D unit cube.
It is known that explicitly computing the volume of this polytope can become too computationally complex in many instances.
However, our algorithm was able to sample $1,000,000$  uniform at random points from the feasible activation space. 
The computed  distribution of activation across  muscles sheds light onto the structure of these solution spaces---rather than simply exploring their maximal and minimal values. 
Although this paper presents a 7 dimensional case of the index finger, our methods extend to systems with at least 40 muscles. This will allow our motor control community  to understand the  distributions of feasible muscle activations, providing important contextual information into learning, optimization and adaptation of motor patterns in future research.
\end{abstract}
%%%%%%%%%%%%%%%%%%%%%%%%%%%%%%%%%%%%%%%%%%%%%%%%%%%%%%%%%%%%%%%%%%%%%%%%%%%%%%%%
\begin{introduction}
\section{INTRODUCTION}

Understanding how the brain selects control strategies  Prior estimations of the feasible activation space (FAS) have examined the lowerbound and upperbound of activation of a muscle.


Below we outline the key ideas of this proceeding:
\begin{itemize}

\item first claim
\item second claim
\item third claim

\end{itemize}

\end{introduction}
\section{Materials and Methods}

\subsection{Polytope representation of the feasible activation space}



\subsection{Hit-and-Run}
The boundaries of the convex polytope defining the feasible activation set are defined by the mechanics of the limb and the constraints of the task, as is described in Subsection \ref{ss:finger}. The goal of the Hit-and-Run algorithm is to uniformly sample  a convex body \cite{smith1984efficient}. 
In the case of a schematic tendon-driven limb with three muscles, the feasible activations start out being the positive unit cube (as muscles can only be activated positively from 0 to a maximal normalized value of 1). As explained in \cite{Valero-Cuevas2009mathematical}, that the feasible activation set for a given static force vector produced at the endpoint of the limb is further reduced by the addition of task constraints  in the form of equality or inequality constraints that define the direction and desired magnitude of the force. Thus if this simple limb is meeting one equality constraint, the feasible activation set os the polygon $P$ containing all feasible activation  $\textbf{a} \in \mathbb{R}^n$ that satisfy
\[\textbf{f} = A\textbf{a}, \textbf{a} \in [0,1]^n,\]
where $\textbf{f} \in \mathbb{R}^m$ is a fixed force vector and $A = J^{-T}RF_m \in \mathbb{R}^{m \times n}$---the matrices of the Jacobian of the limb, the moment arms of the tendons, and the strengths of the muscles, respectively \cite{Valero-Cuevas1998Large,Valero-Cuevas2009mathematical}. $P$ is bounded by the unit $n$-cube since all variables $a_i$, $i \in [n]$ are bounded by 0 and 1 from below, above respectively.
Consider the following $1 \times 3$ fabricated example.
\begin{align*}
&1 = \frac{10}{3}a_1 - \frac{53}{15}a_2 + 2a_3 \\
&a_1, a_2, a_3 \in [0,1],
\end{align*}
the set of feasible activations is given by the shaded set in Figure \ref{fig:fig_hr}.

\begin{figure}[ht]
  \label{fig:fig_hr}
   \begin{center}
    \includegraphics[width=0.25\textwidth]{sections/figs/feasibleactivation.png}
  \end{center}
  \caption{The feasible activation set for a  three-muscle system meeting one functional constraint is a polygon in $\mathbb{R}^3$. Note that muscle activations are assumed to be bounded between $0$ and $1$.}

\end{figure}

The Hit-and-Run walk on $P$ is defined as follows (it works analogously for any convex body). 
\begin{enumerate}
\item Inner Point: Find a given starting point $\textbf{p}$ of $P$ (Figure \ref{fig:hitruncube}(a)) .
\item Direction: Generate a random direction through $\textbf{p}$ (uniformly at random over all directions) (Figure \ref{fig:hitruncube}(b)).
\item Endpoints: Find the intersection points of the random direction with the $n$-unit cube (Figure \ref{fig:hitruncube}(c)).
\item New Point: Choose the next point of the sampling algorithm uniformly at random from the segment of the line in $P$ (Figure \ref{fig:hitruncube}(d)). 
\item Repeat from $(b)$ the above steps with the new point as the starting point .
\end{enumerate}

To find a starting point in 
\[\textbf{f} = A\textbf{a}, \textbf{a} \in [0,1]^n,\]
we only need to find a feasible activation vector. For the Hit-and-Run algorithm to mix faster, we do not want the starting point to be in a vertex of the activation space. We use the following standard trick with slack variables $\epsilon_i$.

\begin{equation}\label{eq:LP_r}
\begin{array}{lrcl}
\mbox{maximize} & \sum_{i=1}^n \epsilon_i \\ 
\mbox{subject to} & \textbf{f} &=& A\textbf{a}\\
  & a_i &\in& [\epsilon_i, 1- \epsilon_i], \hspace{5mm} \forall i \in \{1,\dots,n\}  \\
  & \epsilon_i &\geq& 0, \hspace{5mm} \forall i \in \{1,\dots,n\}.  
\end{array}
\end{equation}

How many steps are necessary to reach a uniformly at random point in the polytope? For convex polygons in higher dimensions c. $40$, experimental results suggest that $\mathcal{O}(n)$ steps of the Hit-and-Run algorithm are sufficient. In particular Emiris and Fisikopoulos paper suggest that $(10 + 10\frac{n})n$ steps are enough to have a close to uniform distribution \cite{emiris2013efficient}.
In the index finger model we executed the Hit-and-Run algorithm $100,000$ times.

\subsection{Realistic index finger model}
\label{ss:finger}
We used our published model in \cite{Valero-Cuevas1998Large} to find matrix $A \in \mathbb{R}^{4 \times 7}$, where $\textbf{a} \in \mathbb{R}^7$ and the four degrees of freedom were ad-abduction and flexion-extension at the metacarpophalangeal joint, and flexion-extension at the proximal and distal interphalangeal joints. The force direction we simulated is that in the palmar direction in the posture shown in Figure \ref{fig:finger}.
\begin{figure}[htbp]
\centering
\includegraphics[width=7.5cm\textwidth]{sections/figs/finger.pdf}
\caption{The index finger model simulated 50\% of maximal force production in the palmar direction. Adapted from \cite{Valero-Cuevas1998Large}.}
\label{fig:finger}
\end{figure}

\begin{figure}[htbp]
\centering
\includegraphics[width=1.0\textwidth]{sections/figs/raw_histograms.png}
\caption{Spatial distribution of muscle activations in the 7-dimensional feasible activation set to produce force in the palmar direction.}
\label{fig:raw_histograms.png}
\end{figure}


\begin{results}

We produce a $$(100000,7)$$ matrix, where each row represents an activation vector lying within the FAS. Each element of the matrix represents a muscle's activation for that given activation vector. Thus, we project the high-dimensional polytope onto each of its 7 dimensions, and visualize the contribution of a given muscle across all of solutions with histogram binning.



\end{results}
\section{DISCUSSION}

Our results clearly show that\\
-The Hit-and-Run algorithm can explore the feasible activation space for a realistic 7-muscle finger in a way that is computationally tractable.\\
-For some muscles, we find that the bounding box exceptionally misconstrues the internal structure of the feasible activation set.\\
-The Hit-and-Run algorithm is cost-agnostic in the sense that no cost function is needed to predict the distribution of muscle activation patterns. Therefore, we can provide spatial context to where 'optimal' solutions lie within the solution space; this approach can be used to explore the consequences of different cost functions.\\
-The distribution of muscle activations often show and strong modes that will critically affect the learning of motor tasks.


If the feasible activation space is skewed or condensed, we can learn about the statistical tendencies of the musculoskeletal system, and better define the plane upon which optimization occurs. This application of Hit-and-Run provides a tool to generate testable hypotheses of how coordination habits may come about, how they are learned, and how difficult or easy it is to break out of them. 




\section{PROCEDURE FOR PAPER SUBMISSION}

\subsection{Selecting a Template (Heading 2)}

First, confirm that you have the correct template for your paper size. This template has been tailored for output on the US-letter paper size. 
It may be used for A4 paper size if the paper size setting is suitably modified.

\subsection{Maintaining the Integrity of the Specifications}

The template is used to format your paper and style the text. All margins, column widths, line spaces, and text fonts are prescribed; please do not alter them. You may note peculiarities. For example, the head margin in this template measures proportionately more than is customary. This measurement and others are deliberate, using specifications that anticipate your paper as one part of the entire proceedings, and not as an independent document. Please do not revise any of the current designations

\section{MATH}

Before you begin to format your paper, first write and save the content as a separate text file. Keep your text and graphic files separate until after the text has been formatted and styled. Do not use hard tabs, and limit use of hard returns to only one return at the end of a paragraph. Do not add any kind of pagination anywhere in the paper. Do not number text heads-the template will do that for you.

$$
\alpha + \beta = \chi \eqno{(1)}
$$


\subsection{Figures and Tables}

Positioning Figures and Tables: Place figures and tables at the top and bottom of columns. Avoid placing them in the middle of columns. Large figures and tables may span across both columns. Figure captions should be below the figures; table heads should appear above the tables. Insert figures and tables after they are cited in the text. Use the abbreviation �Fig. 1�, even at the beginning of a sentence.

\begin{table}[h]
\caption{An Example of a Table}
\label{table_example}
\begin{center}
\begin{tabular}{|c||c|}
\hline
One & Two\\
\hline
Three & Four\\
\hline
\end{tabular}
\end{center}
\end{table}


   \begin{figure}[thpb]
      \centering
      \framebox{\parbox{3in}{We suggest that you use a text box to insert a graphic (which is ideally a 300 dpi TIFF or EPS file, with all fonts embedded) because, in an document, this method is somewhat more stable than directly inserting a picture.
}}
      %\includegraphics[scale=1.0]{figurefile}
      \caption{Inductance of oscillation winding on amorphous
       magnetic core versus DC bias magnetic field}
      \label{figurelabel}
   \end{figure}
  

\addtolength{\textheight}{-12cm}   % This command serves to balance the column lengths
                                  % on the last page of the document manually. It shortens
                                  % the textheight of the last page by a suitable amount.
                                  % This command does not take effect until the next page
                                  % so it should come on the page before the last. Make
                                  % sure that you do not shorten the textheight too much.

%%%%%%%%%%%%%%%%%%%%%%%%%%%%%%%%%%%%%%%%%%%%%%%%%%%%%%%%%%%%%%%%%%%%%%%%%%%%%%%%



%%%%%%%%%%%%%%%%%%%%%%%%%%%%%%%%%%%%%%%%%%%%%%%%%%%%%%%%%%%%%%%%%%%%%%%%%%%%%%%%



%%%%%%%%%%%%%%%%%%%%%%%%%%%%%%%%%%%%%%%%%%%%%%%%%%%%%%%%%%%%%%%%%%%%%%%%%%%%%%%%
\section*{APPENDIX}

NA for now

\section*{ACKNOWLEDGMENT}

@FCO(Insert ETH funding ack.)


%%%%%%%%%%%%%%%%%%%%%%%%%%%%%%%%%%%%%%%%%%%%%%%%%%%%%%%%%%%%%%%%%%%%%%%%%%%%%%%%

References are important to the reader; therefore, each citation must be complete and correct. If at all possible, references should be commonly available publications.



\begin{thebibliography}{99}

\bibitem{c1} G. O. Young, �Synthetic structure of industrial plastics (Book style with paper title and editor),� 	in Plastics, 2nd ed. vol. 3, J. Peters, Ed.  New York: McGraw-Hill, 1964, pp. 15�64.
\bibitem{c2} W.-K. Chen, Linear Networks and Systems (Book style).	Belmont, CA: Wadsworth, 1993, pp. 123�135.
\bibitem{c3} H. Poor, An Introduction to Signal Detection and Estimation.   New York: Springer-Verlag, 1985, ch. 4.
\bibitem{c4} B. Smith, �An approach to graphs of linear forms (Unpublished work style),� unpublished.
\bibitem{c5} E. H. Miller, �A note on reflector arrays (Periodical style�Accepted for publication),� IEEE Trans. Antennas Propagat., to be publised.
\bibitem{c6} J. Wang, �Fundamentals of erbium-doped fiber amplifiers arrays (Periodical style�Submitted for publication),� IEEE J. Quantum Electron., submitted for publication.
\bibitem{c7} C. J. Kaufman, Rocky Mountain Research Lab., Boulder, CO, private communication, May 1995.
\bibitem{c8} Y. Yorozu, M. Hirano, K. Oka, and Y. Tagawa, �Electron spectroscopy studies on magneto-optical media and plastic substrate interfaces(Translation Journals style),� IEEE Transl. J. Magn.Jpn., vol. 2, Aug. 1987, pp. 740�741 [Dig. 9th Annu. Conf. Magnetics Japan, 1982, p. 301].
\bibitem{c9} M. Young, The Techincal Writers Handbook.  Mill Valley, CA: University Science, 1989.
\bibitem{c10} J. U. Duncombe, �Infrared navigation�Part I: An assessment of feasibility (Periodical style),� IEEE Trans. Electron Devices, vol. ED-11, pp. 34�39, Jan. 1959.
\bibitem{c11} S. Chen, B. Mulgrew, and P. M. Grant, �A clustering technique for digital communications channel equalization using radial basis function networks,� IEEE Trans. Neural Networks, vol. 4, pp. 570�578, July 1993.
\bibitem{c12} R. W. Lucky, �Automatic equalization for digital communication,� Bell Syst. Tech. J., vol. 44, no. 4, pp. 547�588, Apr. 1965.
\bibitem{c13} S. P. Bingulac, �On the compatibility of adaptive controllers (Published Conference Proceedings style),� in Proc. 4th Annu. Allerton Conf. Circuits and Systems Theory, New York, 1994, pp. 8�16.
\bibitem{c14} G. R. Faulhaber, �Design of service systems with priority reservation,� in Conf. Rec. 1995 IEEE Int. Conf. Communications, pp. 3�8.
\bibitem{c15} W. D. Doyle, �Magnetization reversal in films with biaxial anisotropy,� in 1987 Proc. INTERMAG Conf., pp. 2.2-1�2.2-6.
\bibitem{c16} G. W. Juette and L. E. Zeffanella, �Radio noise currents n short sections on bundle conductors (Presented Conference Paper style),� presented at the IEEE Summer power Meeting, Dallas, TX, June 22�27, 1990, Paper 90 SM 690-0 PWRS.
\bibitem{c17} J. G. Kreifeldt, �An analysis of surface-detected EMG as an amplitude-modulated noise,� presented at the 1989 Int. Conf. Medicine and Biological Engineering, Chicago, IL.
\bibitem{c18} J. Williams, �Narrow-band analyzer (Thesis or Dissertation style),� Ph.D. dissertation, Dept. Elect. Eng., Harvard Univ., Cambridge, MA, 1993. 
\bibitem{c19} N. Kawasaki, �Parametric study of thermal and chemical nonequilibrium nozzle flow,� M.S. thesis, Dept. Electron. Eng., Osaka Univ., Osaka, Japan, 1993.
\bibitem{c20} J. P. Wilkinson, �Nonlinear resonant circuit devices (Patent style),� U.S. Patent 3 624 12, July 16, 1990. 






\end{thebibliography}




\end{document}
