\section{DISCUSSION}

Our results clearly show that
-We find that a the Hit-and-Run algorithm can explore the feasible activation space for a realistic 7-muscle finger in a way that is computationally tractable.\\
-For some muscles, we find that the bounding box exceptionally misconstrues the actual shape of the feasible activation space.\\
-The Hit-and-Run algorithm is cost-agnostic in the sense that no cost function is needed to predict muscle activation patterns. Therefore, we can provide spatial context to where 'optimal' solutions lie within the space; this approach can be used to view where local optima exist.\\
-The distribution of muscle activations often show asymmetry, and strong modes.


