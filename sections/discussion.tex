\section{DISCUSSION}

Our results clearly show that\\
-The Hit-and-Run algorithm can explore the feasible activation space for a realistic 7-muscle finger in a way that is computationally tractable.\\
-For some muscles, we find that the bounding box exceptionally misconstrues the internal structure of the feasible activation set.\\
-The Hit-and-Run algorithm is cost-agnostic in the sense that no cost function is needed to predict the distribution of muscle activation patterns. Therefore, we can provide spatial context to where 'optimal' solutions lie within the solution space; this approach can be used to explore the consequences of different cost functions.\\
-The distribution of muscle activations often show and strong modes that will critically affect the learning of motor tasks.


If the feasible activation space is skewed or condensed, we can learn about the statistical tendencies of the musculoskeletal system, and better define the plane upon which optimization occurs. This application of Hit-and-Run provides a tool to generate testable hypotheses of how coordination habits may come about, how they are learned, and how difficult or easy it is to break out of them. 


