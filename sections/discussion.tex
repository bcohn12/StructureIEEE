\section{DISCUSSION}

Our results and methodology raise the following ideas:\\
{\begin{itemize}
	\item The Hit-and-Run algorithm can explore the feasible activation space for a realistic 7-muscle finger in a way that is computationally tractable.
	\item For some muscles, we find that the bounding box exceptionally misconstrues the internal structure of the feasible activation set.
	\item The Hit-and-Run algorithm is cost-agnostic in the sense that no cost function is needed to predict the distribution of muscle activation patterns. Therefore, we can provide spatial context to where `optimal' solutions lie within the solution space; this approach can be used to explore the consequences of different cost functions.
	\item The distribution of muscle activations may be intricately related to strong modes which critically affect the learning of motor tasks.
\end{itemize}}
With the spatial context of the feasible activation space, we can explore the statistical tendencies of a musculoskeletal system, and better define the landscape upon which optimization occurs. This application of Hit-and-Run provides a tool to generate testable hypotheses of how coordination habits may come about, how they are learned, and how difficult or easy it is to break out of them. 




