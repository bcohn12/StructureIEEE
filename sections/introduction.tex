
\section{INTRODUCTION}

Muscle redundancy is the term used to describe the underdetermined nature of neural control of musculature.
The classical notion of muscle redundancy  proposes that, faced with an infinite number of possible muscle activation patterns for a given task, the nervous system uses optimization to select a given specific solution.
Here, each of the $N$ muscles represents a dimension of control, and a muscle activation pattern is a point in $\mathbb{R}^N$ \cite{Valero-Cuevas1998Large}.
Thus researchers often seek to infer the optimization approach and the cost functions the nervous system likely utilizes to find the points in activation space to produce natural behavior \cite{Chao1978Graphical,Prilutsky2000Muscle,scott2004optimal,todorov2002optimal,crowninshield1981physiologically,higginson2005simulated}. 


Implicit in these optimization procedures is the notion that there exists a well structured set of feasible solutions. Thus several of us have focused on describing and understanding those high-dimensional subspaces  embedded in $\mathbb{R}^N$ \cite{kutch2011muscle,kutch2012challenges,sohn2013cat_bounding_box,Valero-Cuevas1998Large,Valero-Cuevas2015high-dimensional}.

For the case of muscle redundancy for submaximal and static force production with a limb,  the problem is phrased as one of computational geometry: find the convex polytope of feasible muscle activations given the mechanics of the limb and the constrains of the task \cite{avis1992Pivoting,Valero-Cuevas1998Large,Valero-Cuevas2009mathematical,Valero-Cuevas2015high-dimensional}.  This convex polytope is called the \emph{feasible activation set}. To date, the structure of this high-dimensional polytope is inferred by its bounding box  \cite{kutch2011muscle,sohn2013cat_bounding_box,Valero-Cuevas2015high-dimensional}.  But the bounding box of a convex polytope will always overestimate its volume, and lose the details of its shape.  Empirical dimensionality-reduction methods have also been used to calculate a basis vectors for such subspaces \cite{Clewley2008Estimating,davella2005shared,krishnamoorthy2003muscle}. But those basis  vectors only provide a description of the dimension, orientation, and aspect ratio of the polytope, but not of its boundaries or internal  structure.

Here we present a novel application of the well-known Hit-and-Run algorithm \cite{smith1984efficient} to describe the internal structure of these high-dimensional feasible activation sets. We apply our technique to a schematic example with three muscles to describe the method, and then use realistic model of an index finger with seven muscles and four joints \cite{Valero-Cuevas1998Large}.