\section{Materials and Methods}

\subsection{Polytope representation of the feasible activation space}



\subsection{Hit-and-Run}
The boundaries of the convex polytope defining the feasible activation set are defined by the mechanics of the limb and the constraints of the task, as is described in Subsection \ref{ss:finger}. The goal of the Hit-and-Run algorithm is to uniformly sample  a convex body \cite{smith1984efficient}. 
In the case of a schematic tendon-driven limb with three muscles, the feasible activations start out being the positive unit cube (as muscles can only be activated positively from 0 to a maximal normalized value of 1). As explained in \cite{Valero-Cuevas2009mathematical}, that the feasible activation set for a given static force vector produced at the endpoint of the limb is further reduced by the addition of task constraints  in the form of equality or inequality constraints that define the direction and desired magnitude of the force. Thus if this simple limb is meeting one equality constraint, the feasible activation set os the polygon $P$ containing all feasible activation  $\textbf{a} \in \mathbb{R}^n$ that satisfy
\[\textbf{f} = A\textbf{a}, \textbf{a} \in [0,1]^n,\]
where $\textbf{f} \in \mathbb{R}^m$ is a fixed force vector and $A = J^{-T}RF_m \in \mathbb{R}^{m \times n}$---the matrices of the Jacobian of the limb, the moment arms of the tendons, and the strengths of the muscles, respectively \cite{Valero-Cuevas1998Large,Valero-Cuevas2009mathematical}. $P$ is bounded by the unit $n$-cube since all variables $a_i$, $i \in [n]$ are bounded by 0 and 1 from below, above respectively.
Consider the following $1 \times 3$ fabricated example.
\begin{align*}
&1 = \frac{10}{3}a_1 - \frac{53}{15}a_2 + 2a_3 \\
&a_1, a_2, a_3 \in [0,1],
\end{align*}
the set of feasible activations is given by the shaded set in Figure \ref{fig:fig_hr}.

\begin{figure}[ht]
  \label{fig:fig_hr}
   \begin{center}
    \includegraphics[width=0.25\textwidth]{sections/figs/feasibleactivation.png}
  \end{center}
  \caption{The feasible activation set for a  three-muscle system meeting one functional constraint is a polygon in $\mathbb{R}^3$. Note that muscle activations are assumed to be bounded between $0$ and $1$.}

\end{figure}

The Hit-and-Run walk on $P$ is defined as follows (it works analogously for any convex body). 
\begin{enumerate}
\item Inner Point: Find a given starting point $\textbf{p}$ of $P$ (Figure \ref{fig:hitruncube}(a)) .
\item Direction: Generate a random direction through $\textbf{p}$ (uniformly at random over all directions) (Figure \ref{fig:hitruncube}(b)).
\item Endpoints: Find the intersection points of the random direction with the $n$-unit cube (Figure \ref{fig:hitruncube}(c)).
\item New Point: Choose the next point of the sampling algorithm uniformly at random from the segment of the line in $P$ (Figure \ref{fig:hitruncube}(d)). 
\item Repeat from $(b)$ the above steps with the new point as the starting point .
\end{enumerate}

To find a starting point in 
\[\textbf{f} = A\textbf{a}, \textbf{a} \in [0,1]^n,\]
we only need to find a feasible activation vector. For the Hit-and-Run algorithm to mix faster, we do not want the starting point to be in a vertex of the activation space. We use the following standard trick with slack variables $\epsilon_i$.

\begin{equation}\label{eq:LP_r}
\begin{array}{lrcl}
\mbox{maximize} & \sum_{i=1}^n \epsilon_i \\ 
\mbox{subject to} & \textbf{f} &=& A\textbf{a}\\
  & a_i &\in& [\epsilon_i, 1- \epsilon_i], \hspace{5mm} \forall i \in \{1,\dots,n\}  \\
  & \epsilon_i &\geq& 0, \hspace{5mm} \forall i \in \{1,\dots,n\}.  
\end{array}
\end{equation}

How many steps are necessary to reach a uniformly at random point in the polytope? For convex polygons in higher dimensions c. $40$, experimental results suggest that $\mathcal{O}(n)$ steps of the Hit-and-Run algorithm are sufficient. In particular Emiris and Fisikopoulos paper suggest that $(10 + 10\frac{n})n$ steps are enough to have a close to uniform distribution \cite{emiris2013efficient}.
In the index finger model we executed the Hit-and-Run algorithm $100,000$ times.

\subsection{Realistic index finger model}
\label{ss:finger}
We used our published model in \cite{Valero-Cuevas1998Large} to find matrix $A \in \mathbb{R}^{4 \times 7}$, where $\textbf{a} \in \mathbb{R}^7$ and the four degrees of freedom were ad-abduction and flexion-extension at the metacarpophalangeal joint, and flexion-extension at the proximal and distal interphalangeal joints. The force direction we simulated is that in the palmar direction in the posture shown in Figure \ref{fig:finger}.
\begin{figure}[htbp]
\centering
\includegraphics[width=7.5cm\textwidth]{sections/figs/finger.pdf}
\caption{The index finger model simulated 50\% of maximal force production in the palmar direction. Adapted from \cite{Valero-Cuevas1998Large}.}
\label{fig:finger}
\end{figure}

\begin{figure}[htbp]
\centering
\includegraphics[width=1.0\textwidth]{sections/figs/raw_histograms.png}
\caption{Spatial distribution of muscle activations in the 7-dimensional feasible activation set to produce force in the palmar direction.}
\label{fig:raw_histograms.png}
\end{figure}

