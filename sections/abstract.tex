\begin{abstract}
The brain must select its control strategies among an infinite set of possibilties, thereby solving an optimization problem. 
While this set is infinite and lies in high dimensions, it is bounded by kinematic, neuromuscular, and anatomical constraints, within which the brain must select optimal solutions. 
We use data from a human index finger with 7 muscles, 4DOF, and 4 output dimensions. For a given force vector at the endpoint, the feasible activation space is a 3D convex polytope, embedded in the 7D unit cube.
It is known that explicitly computing the volume of this polytope can become too computationally complex in many instances. 
We generated random points in the feasible activation space using the Hit-and-Run method, which converged to the uniform distribution. 
After generating enough points, we computed the distribution of activation across each muscle, shedding light onto the structure of these solution spaces- rather than simply exploring their maximal and nimimal values. 
We also visualize the change in these activation distributions as we march toward maximal feasible force production in a given direction. 
Using the parallel coordintes method, we visualize the connection between the muscle activations. Once can then explore the feasible activation space, while constraining certain muscles.
Although this paper presents a 7 dimensional case of the index finger, our methods extend to systems with up to at least 40 muscles. We challenge the community to map the shapes distributions of each variable in the solution space, thereby providing important contextual information into optimization of motor cortical function in future research.
\end{abstract}